\section{Introduction}

\emph{Algorithm portfolio} methods \cite{Huberman1997Economics} use information
about solvers and problem instances to allocate computational resources among
multiple solvers, attempting to maximize the time spent on those well suited to
each instance. Portfolio methods such as SATzilla \cite{Xu2008SATzilla} have
proved increasingly effective in satisfiability.

An algorithm portfolio must decide which solvers to run and for how long to run
them. These decisions rely entirely on expectations about solver behavior.

The {\tt borg-sat} solver attempts to to learn predictable aspects of solver
behavior---such as how likely a solver is to succeed if it has previously
failed---given data on the successes and failures of solvers on many problem
instances. The version of this solver submitted to SAT-Race 2010, {\tt
borg-sat-10.06.07}, assumes a specific \emph{latent class} model of solver
behavior, a mixture of Dirichlet compound multinomial (DCM) distributions,
which is used to identify groups of similar problem instances. This model is
examined in detail by Silverthorn and Miikkulainen
(2010)\nocite{Silverthorn2010latent}. It captures the basic correlations
between solvers, runs, and problem instances, as well as the tendency of solver
outcomes to recur. Unlike the classifier employed by SATzilla, the model
considers only the success or failure of each past solver run; it does
\emph{not} consider instance feature information.

This version of {\tt borg-sat} employs the DCM mixture
model in computing an optimal fixed-length solver execution schedule followed
for every problem instance, as described in the following section.

