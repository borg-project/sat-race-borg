\section{\label{sec:subsolvers}Portfolio Composition}

Portfolio methods rely entirely on the performance of the solvers they employ,
and are possible only because of the engineering and research involved in
making those solvers effective. This version of {\tt borg-sat} considered 13
subsolvers in its model: every qualifying solver in the application category of
the final round of the 2009 SAT competition, excluding the reference solvers
and SATzilla, with two exceptions ({\tt kw} and {\tt MiniSat 2.1}), as well as
two more recent solvers ({\tt cryptominisat-2.4.2} and {\tt
precosat-465r2-2ce82ba-100514}). Table \ref{tab:subsolvers} lists these solvers
and their authors. Note that not all of these solvers were necessarily included
in the final execution plan.

\begin{table*}
\begin{center}
\begin{tabular}{lll}
\toprule
{\bf Name}         & {\bf Reference}\\
\midrule
{\tt CircUs 2009-03-23}             & Hyojung Han\\
{\tt clasp 1.2.0-SAT09-32}          & Benjamin Kaufmann\\
{\tt glucose 1.0}                   & Gilles Audemard and Laurent Simon\\
{\tt LySAT i/2009-03-20}            & Youssef Hamadi, Sa\"{i}d Jabbour, and Lakhdar Sa\"{i}s\\
{\tt ManySAT 1.1 aimd 1/2009-03-20} & Youssef Hamadi, Sa\"{i}d Jabbour, and Lakhdar Sa\"{i}s\\
{\tt MiniSAT 09z 2009-03-22}        & Markus Iser\\
{\tt minisat\_cumr p-2009-03-18}    & Kazuya Masuda and Tomio Kamada\\
{\tt MXC 2009-03-10}                & David Bregman\\
{\tt precosat 236}                  & Armin Biere\\
{\tt Rsat 2009-03-22}               & Knot Pipatsrisawat and Adnan Darwiche\\
{\tt SApperloT base}                & Stephan Kottler\\
{\tt cryptominisat-2.4.2}           & Mate Soos\\
{\tt precosat-465r2-2ce82ba-100514} & Armin Biere\\
\bottomrule
\end{tabular}
\caption{\label{tab:subsolvers}Subsolvers considered by the {\tt
borg-sat-10.06.07} planner.}
\end{center}
\end{table*}

